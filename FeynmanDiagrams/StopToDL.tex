
%%%%%%%%%%%%%%%%%%%%%% Feynman diagram for StopToDL

\documentclass{article}

\input{shared/header.tex}

\def\MainQuark{d}


%%%%%%%%%%%%%%%%%%%%%%%%%%% Document %%%%%%%%%%%%%%%%%%%%%%%%%%%
\begin{document}
\thispagestyle{empty}


%%%%%%%% THE NAME OF THE fmffile HAS TO BE ``Feynman<filename>'' TO USE compile.py %%%%%%%%%%%%%%%
\begin{fmffile}{FeynmanStopToDL}
\parbox{300mm}{

\begin{fmfgraph*}(180,90) %\fmfpen{thick}
  \fmfset{arrow_len}{cm}\fmfset{arrow_ang}{0}
  
  %%%%%%%%%%%% Specifying number of inputs/outputs
  \fmfleftn{i}{2}
  \fmfrightn{o}{4}
  \fmflabel{}{i1}
  \fmflabel{}{i2}
    
  %%%%%%%%%%%% Incoming protons (one line)
  \fmf{fermion,  tension=2., lab=p, label.side=right}{v1,i1}
  \fmf{fermion,  tension=2., lab=p, label.side=left}{v1,i2}
           
  %%%%%%%%%%%% Produced SUSY particles
  \fmf{dashes, label=\sTop, label.side=left}{v1,v3}
  \fmf{dashes, label=\sTopb, label.side=right}{v1,v2}

  %%%%%%%%%%%%% Decays and vertex circles
  \fmf{fermion}{v3,o4} 
  \fmflabel{\MainQuark}{o4}

  \fmf{fermion}{v3,o3}
  \fmflabel{$\ell^{-}$}{o3}

  %% 2nd decay
  \fmf{fermion}{v2,o1}
  \fmflabel{$\mathrm{\overline{\MainQuark}}$}{o1}
 
  \fmf{fermion}{v2,o2}
  \fmflabel{$\ell^{+}$}{o2}

  %% Vertex circles
  \fmfdot{v2,v3}
           
  %%%%%%%%%%%% Additional lines on incoming protons and blob
  \input{shared/protons.tex}

\end{fmfgraph*}
       
}           
\end{fmffile} 

\end{document}
